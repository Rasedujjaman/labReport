\section*{Data Collection}
\addcontentsline{toc}{section}{Data Collection}
	
	
	
	\begin{center}
		\begin{tabular}{l c}
			\toprule
			\textbf{Condition} & \textbf{Current through $R_2$ (mA)} \\
			\midrule
			$V_1$ and $V_2$ active &  \\
			Only $V_1$ active &  \\
			Only $V_2$ active &  \\
			\bottomrule
		\end{tabular}
	\end{center}
	
	\section*{Calculations}
	
	Given:
	\[
	R_1 = 3.3\,k\Omega,\quad R_2 = 1\,k\Omega,\quad R_3 = 4.7\,k\Omega
	\]
	\[
	V_1 = 10\,V,\quad V_2 = 5\,V
	\]
	
	\subsection*{Case 1: Only $V_1$ Active}
	$V_2$ is replaced by a short circuit.  
	Using nodal analysis, the current through $R_2$ is calculated as $I_{R_2(V_1)}$.
	
	\subsection*{Case 2: Only $V_2$ Active}
	$V_1$ is replaced by a short circuit.  
	The current through $R_2$ is calculated as $I_{R_2(V_2)}$.
	
	\subsection*{Case 3: Both Sources Active}
	The total current through $R_2$ is:
	\[
	I_{R_2} = I_{R_2(V_1)} + I_{R_2(V_2)}
	\]