\section*{Discussion and Conclusion}
\addcontentsline{toc}{section}{Discussion and Conclusion}

The experiment successfully demonstrated the validity of the superposition theorem for a linear resistive network containing two independent voltage sources. The calculated node voltage using superposition closely matched the value obtained from direct nodal analysis, confirming that the total response in a linear circuit is indeed the sum of the individual responses produced by each independent source acting alone.
	
	The individual contributions from $V_1$ and $V_2$ were first determined by turning off one source at a time and observing the effect on the current through $R_2$. When the contributions were added, the resulting current and node voltage closely matched the values obtained when both sources were active simultaneously. This agreement verifies that linearity holds for the given circuit.
	
	Minor deviations that may occur in practical measurements can be attributed to resistor tolerances, source internal resistance, measurement instrument precision, and wiring losses. Nevertheless, such discrepancies do not significantly affect the overall verification of the theorem.
	
	In conclusion, the experiment confirms that the superposition theorem accurately predicts circuit behavior in linear DC circuits. The results obtained reinforce the theoretical expectation that complex circuits with multiple sources can be analyzed by isolating each source and combining their individual effects.
	